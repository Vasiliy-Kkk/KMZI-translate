\section{Необходимый материал}
\subsection{Обозначения и основные определения}
\textbf{Конечные поля.} Существует, с точностью до изоморфизмов, единственное конечное поле, состоящее из \(2^n\) элементов, которое обозначается \(\text{GF}(2^n)\). В данной работе будет использоваться "\(\oplus\)" для обозначения сложения в поле, а \(a \odot b\) или \(ab\) для обозначения произведения \(a, b \in \text{GF}(2^n)\). Для всех \(x \in \text{GF}(2^n)\) выполняется \(x^{2^n} \oplus x = 0\).

Пусть \(n = 2m\), тогда определим отображение из \(\text{GF}(2^{2m})\) в \(\text{GF}(2^m)\) как функцию \(\text{Tr}_m: \text{GF}(2^{2m}) \rightarrow \text{GF}(2^m)\), такую что \(\text{Tr}_m(x) = x^{2^m} \oplus x\). Для любого \(\beta \in \text{GF}(2^{2m})\), такого что \(\text{Tr}_m(\beta) = 1\), элементы из \(\text{GF}(2^{2m})\) могут быть единственным образом разложены в виде \(a\beta \oplus b\), где \(a\) и \(b\) принадлежат \(\text{GF}(2^m)\). Биекция, отображающая \(x \in \text{GF}(2^{2m})\) в \((a, b) \in \text{GF}(2^m)\)\footnote{Строго говоря, \(GF(2^4)^*\) не является аддитивной подгруппой \(GF(2^8)\). Таким образом, множество \(\{a \oplus x, x \in GF(2^4)^*\}\) формально не является аддитивным смежным классом. Однако, для простоты будем немного злоупотреблять этим термином и называть такие множества "аддитивными смежными классами \(GF(2^4)^*\)".} такая, что \(x = a\beta \oplus b\), обозначается как \(\text{Split}_\beta\), таким образом \(\text{Split}_\beta(a\beta \oplus b) = (a, b)\). Это линейная функция.

Будем использовать \(S^*\) для обозначения множества \(S\) без нуля. Хотя \(\text{GF}(2^m)^*\) не является аддитивной группой, будем называть \(\{a \oplus x, x \in \text{GF}(2^m)^*\}\) аддитивным смежным классом \(\text{GF}(2^m)^*\) для упрощения. Пусть \(S\) — это множество, \(a\) — константа, тогда обозначим \(a \oplus S = \{a \oplus x, x \in S\}\) и \(aS = a \odot S = \{a \odot x, x \in S\}\).

\textbf{Двоичные последовательности как элементы кольца.} Полем \(\text{GF}(2^n)\) называется \(\mathbb{F}_2[X]/p(X)\) для некоторого неприводимого полинома \(p\) степени \(n\). Если \(\alpha\) — это корень \(p\), то все элементы \(\text{GF}(2^n)\) можно представить в виде \(\sum_{i=0}^{n-1} x_i \alpha^i\), где \(x_i \in \mathbb{F}_2\). Тогда двоичную последовательность \((x_0, \ldots, x_{n-1}) \in \mathbb{F}_2^n\) можно представить в виде \(\sum_{i=0}^{n-1} x_i \alpha^i \in \text{GF}(2^n)\) аналогично тому, как ее можно представить в виде \(\sum_{i=0}^{n-1} x_i 2^i \in \mathbb{Z}/2^n\mathbb{Z}\). В этом случае, двоичное представление \(a \oplus b\) для \(a, b \in \text{GF}(2^n)\) является исключающим ИЛИ (XOR) двоичных представлений \(a\) и \(b\).

\textbf{Логарифмы.} Для любого \(x \in \text{GF}(2^n)^*\), логарифм \(\log_\alpha(x)\) — это целое число из \(\mathbb{Z}/(2^n - 1)\mathbb{Z}\) такое, что \(\alpha^{\log_\alpha(x)} = x\). Такая функция не является перестановкой, потому что она не определена в 0. Эта проблема может быть решена различными способами. В \cite{HN10} Хакала и Ниберг изучают функцию, которая в этой работе будет обозначаться как \(\log_{\alpha}^{\text{HN}}\), в то время как Фэнг и др. в \cite{FLY09} предложили другую вариацию, частный случай которой является перестановкой \(\text{GF}(2^n)\), и которая будет обозначаться как \(\log_{\alpha}^{\text{FLY}}\). Эти две функции отображают \(\text{GF}(2^n)\) в \(\mathbb{Z}/2^n\mathbb{Z}\) и определены следующим образом:
$$
\log _\alpha^{\mathrm{HN}}(x)=\left\{\begin{array}{ll}
        2^n-1 & \text { если } x=0, \\
        0 & \text { если } x=1, \\
        \log _\alpha(x) & \text { если } x \notin\{0,1\},
        \end{array} \quad \text { и } \log _\alpha^{\mathrm{FLY}}(x)= \begin{cases}0 & \text { если } x=0 \\
        2^n-1 & \text { если } x=1, \\
        \log _\alpha(x) & \text { если } x \notin\{0,1\} .\end{cases}\right.
$$

\textbf{Булевы функции.} Пусть \( F : \mathbb{F}_2^n \rightarrow \mathbb{F}_2^m \) — функция. Таблица линейных аппроксимаций (LAT) или преобразование Уолша функции \( F \) представляет собой матрицу размером \( 2^n \times 2^m \), обозначаемую как \( \mathcal{W}_F \), для которой:

\[
\mathcal{W}_F(a, b) = \sum_{x \in \mathbb{F}_2^n} (-1)^{a \cdot x \oplus b \cdot F(x)},
\] где \( a \cdot b \) — это обычное скалярное произведение в \(\mathbb{F}_2^n\). Максимальное значение \(|\mathcal{W}_F(a, b)|\) для \(b \neq 0\) называется линейностью \( F \). Таблица распределения разностей (Difference Distribution Table, DDT) функции \( F \) — это матрица \( 2^n \times 2^m \), обозначаемая \(\delta_F\), такая что
\[
\delta_F(a, b) = \# \{ x \in \mathbb{F}_2^n \mid F(x \oplus a) \oplus F(x) = b \}.
\]

Максимальное значение \(\delta_F(a, b)\) для \(a \neq 0\) yfpsdftncz hfpyjcnyjq однородностью \( F \). Если \( A \) и \( B \) — аффинные перестановки над \(\mathbb{F}_2^m\) и \(\mathbb{F}_2^n\) соответственно, то \( F \) аффинно эквивалентна \( G = B \circ F \circ A \). Кроме того, пусть \( L_A \) и \( L_B \) — линейные составляющие \( A \) и \( B \). Тогда LAT функции \( G \) вычисляется как:

\[
\mathcal{W}_G(a, b) = \mathcal{W}_F((L_A^{-1})^T(a), L_B^T(b)).
\]

Аффинно-эквивалентность может быть обобщена в CCZ-эквивалентность \cite{CCZ98}: две функции \( F: \mathbb{F}_2^n \rightarrow \mathbb{F}_2^m \) и \( G: \mathbb{F}_2^n \rightarrow \mathbb{F}_2^m \) являются CCZ-эквивалентными, если существует аффинная перестановка \( \mathcal{A} \) множества \(\mathbb{F}_2^n \times \mathbb{F}_2^m\) такая, что

\[
\{(x, F(x)) \mid x \in \mathbb{F}_2^n \} = \mathcal{A}(\{(x, G(x)) \mid x \in \mathbb{F}_2^n \}).
\]

Эта форма эквивалентности известна тем, что сохраняет, в том числе, разностную однородность и линейность.