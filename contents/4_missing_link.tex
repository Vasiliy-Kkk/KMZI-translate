\section{The Missing Link}
\label{sec:missing_link}
Структура TKlog является недостающим звеном между двумя предыдущими разложениями \(\pi\). Её связь с разложениями на основе логарифмов \cite{PU16} естественна, так как обе состоят из варианта дискретного логарифма, за которым следует некоторая арифметика. Факт, что \(\pi\) имеет TU-разложение, остаётся априори удивительным. Однако в разделе~\ref{sec:TU_decomposition_always_exists} мы показываем, что это всегда так для TKlog. Мы также перечисляем некоторые последствия этого свойства в разделе~\ref{sec:consequences}.

\subsection{A TU-Decomposition Always Exists}
\label{sec:TU_decomposition_always_exists}
TKlog всегда может быть выражен схожим образом с первым разложением Бирюкова и др. \cite{BPU16a}. Для его установления мы выводим следующее лемма.

\begin{lemma}
Существует функция \(\gamma : (\mathbb{Z}/(2m + 1)\mathbb{Z})^* \to \mathbb{Z}/(2m - 1)\mathbb{Z}\) такая, что 
\[
\{x \in \mathbb{GF}(2^{2m}), \mathrm{Tr}(x) = 1\} = \{\alpha^i + (2m+1)\gamma(i), i \in \mathbb{Z}/(2m + 1)Z^*\}.
\]
\end{lemma}

\begin{proof}
Пусть \((a, b) = \mathrm{Split}_{\beta}(x)\). Если \(x \notin \mathbb{GF}(2^m)\), то \(a \neq 0\) и можно записать \(x = a(\beta \oplus c)\), где \(c = b/a\) является элементом \(\mathbb{GF}(2^m)\), таким образом, что \(\mathrm{Tr}(\beta \oplus c) = \mathrm{Tr}(\beta) = 1\). Так как такое разложение существует для всех \(x \notin \mathbb{GF}(2^m)\), мы делаем вывод, что множество 
\[
\{\mathrm{logFLY}_{\alpha}(\gamma \oplus \beta), \gamma \in \mathbb{GF}(2^m)\}
\]
должно содержать представителя каждого класса эквивалентности по модулю \(2m + 1\), так как в противном случае некоторые элементы \(\alpha^i + (2m+1)j\) при \(i \neq 0\) не могли бы быть записаны в виде \(a(\beta \oplus c)\). Соответственно,
\[
\{\mathrm{logFLY}_{\alpha}(\gamma \oplus \beta) \mod (2m + 1), \gamma \in \mathbb{GF}(2^m)\} = \{1, \ldots, 2m\},
\]
и, следовательно, должна существовать функция \(\gamma\), как описано выше, такая что
\[
\{\mathrm{logFLY}_{\alpha}(\gamma \oplus \beta), \gamma \in \mathbb{GF}(2^m)\} = \{1 + (2m + 1)\gamma(1), \ldots, 2m + (2m + 1)\gamma(2m)\}.
\]
Лемма следует из этого.
\end{proof}

\begin{theorem}[TU-Decomposition of the TKlog]
Перестановка \(T_{\kappa, s}\) над \(\mathbb{GF}(2^{2m})\) имеет TU-разложение, включающее три \(m\)-битные перестановки \(\tau, \nu\) и \(\sigma\), и \(m\)-битную функцию \(\varphi\). Это представлено в Алгоритме~\ref{alg:TU_decomposition}.
\end{theorem}

\begin{proof}
Пусть \((a, b) = \mathrm{Split}_{\beta}(x)\). Интуитивно, \(a\) будет соответствовать правой ветви TU-разложения, а \(b/a\) левой. В частности, случай \(a = 0\) является особым случаем. Это доказательство является прямым в том смысле, что оно "просто" заключается в правильном определении подкомпонентов \(\tau, \sigma, \nu\) и \(\varphi\) и проверке их работы.

TODO: листинг

Если \(a = 0\), то тогда \(x \in \mathbb{GF}(2^m)\), и определение TKlog приводит к \(T_{\kappa, s}(x) = \kappa(\tau(b))\), где
\[
\tau : 
\begin{cases}
\mathbb{GF}(2^m) \to \mathbb{F}^m_2 \\
0 \mapsto 0, \\
b \mapsto 2^m - \mathrm{logFLY}_{(\alpha^{2m+1})}(b), \text{ если } b \neq 0.
\end{cases}
\]
Действительно, пусть \(x = b = \alpha^{(2m+1)j}\) при \(0 < j \leq 2m - 1\). Для \(b \neq 0, 1\) имеем, что \(\mathrm{logFLY}_{(\alpha^{2m+1})}(b) = \mathrm{log}_{\alpha}(b)/(2m + 1) = j\), так что \(\kappa(\tau(b))\) действительно равно \(T_{\kappa, s}(x)\). Для \(b = 0\), мы тоже сразу получаем, что \(\kappa(\tau(b)) = T_{\kappa, s}(x)\). Для \(b = 1\), \(\mathrm{logFLY}_{\alpha}(b) = 2^{2m} - 1 = (2m - 1)(2m + 1)\), так что \(j = 2m - 1\), что действительно равно \(\mathrm{logFLY}_{(\alpha^{2m+1})}(b)\).

\(\phi \neq 0\). Однако, так как \(\alpha \neq 0\), мы можем записать \(x = \alpha(\beta \oplus b/\alpha)\) и, поскольку 0 и 1 находятся в \(\mathbb{GF}(2^m)\), мы имеем \(x \neq 0, 1\), так что \(\mathrm{logFLY}_{\alpha}(x) = \mathrm{log}_{\alpha}(x)\). Для оценки \(T_{\kappa,s}(x)\), нам нужно найти \(i\) и \(j\) такие, что \(\alpha(\beta \oplus b/\alpha) = \alpha^{i+(2m+1)j}\). Имеется \(\mathrm{Tr}_m(\alpha) = 0\) и \(\mathrm{Tr}_m(\beta \oplus b/\alpha) = 1\), так что мы можем применить Лемму~\ref{lemma:gamma_function} и записать
\[
\begin{cases}
\beta \oplus b/\alpha = \alpha^{i+(2m+1)\gamma(i)} \\
\alpha = \alpha^{(2m+1)(j-\gamma(i))}
\end{cases}
\]

Мы определяем перестановки \(\nu, \sigma\) и функцию \(\varphi\) следующим образом:

\begin{itemize}
  \item Перестановка \(\nu\) захватывает то, как логарифм и арифметическая операция \(x \mapsto 2^m - x\) работают с \(b/\alpha\) для корректного ввода для \(\kappa\):
  \[
  \nu :
  \begin{cases}
  \mathbb{GF}(2^m) \to \mathbb{F}^4_2 \\
  c \mapsto 2^m - (\mathrm{log}_{\alpha}(\beta \oplus c) \mod (2m + 1)),
  \end{cases}
  \]
  так что \(\nu(b/\alpha) = 2^m - i\).

  \item Перестановка \(\sigma\) соответствует перестановке \(s\), примененной к \(j\):
  \[
  \sigma :
  \begin{cases}
  \mathbb{GF}(2^m) \to \mathbb{GF}(2^m) \\
  0 \mapsto 0 \\
  c \mapsto (\alpha^{2m+1}) s(\mathrm{log}_{(\alpha^{2m+1})}(c)), \text{ когда } c \neq 0.
  \end{cases}
  \]

  \item Функция \(\varphi\) соответствует функции \(\gamma\), введенной благодаря Лемме~\ref{lemma:gamma_function}. Она определена следующим образом:
  \[
  \varphi :
  \begin{cases}
  \mathbb{F}^m_2 \to \mathbb{GF}(2^m) \\
  i \mapsto \alpha^{(2m+1)\gamma(2m-i)},
  \end{cases}
  \]
  где \(i \in \mathbb{F}^m_2\) интерпретируется как элемент \(\mathbb{Z}/(2m + 1)\mathbb{Z}\). Заметьте, что \(\varphi(i) \neq 0\) для всех \(i\). Используя, что \(\nu(b/\alpha) = 2^m - i\), имеем
  \[
  (\varphi \circ \nu)(b/\alpha) = \varphi(2^m - i) = \alpha^{(2m+1)\gamma(i)}
  \]

  и, также используя уравнение~\ref{eq:lemma_3}, получаем
  \[
  (\varphi \circ \nu)(b/\alpha) \cdot \alpha = \alpha^{(2m+1)\gamma(i)} \cdot \alpha^{(2m+1)(j-\gamma(i))} = \alpha^{(2m+1)j}.
  \]

  Соединяя это результат с определением \(\sigma\), мы получаем
  \[
  \sigma((\varphi \circ \nu)(b/\alpha) \cdot \alpha) = \sigma(\alpha^{(2m+1)j}) = (\alpha^{2m+1}) s(j).
  \]

\end{itemize}
Мы можем затем записать для любого \(\alpha \neq 0\), что
\[
T_{\kappa, s}(\alpha \beta \oplus b) = \kappa(\nu(b/\alpha)) \oplus \sigma((\varphi \circ \nu)(b/\alpha) \cdot \alpha).
\]

Когда \(\alpha = 0\), термин справа от \(\oplus\) исчезает, так как \(\sigma(0) = 0\). Поэтому, когда \(\alpha = 0\), можем записать:
\[
T_{\kappa, s}(\alpha \beta \oplus b) = \kappa(\tau(b)) \oplus \sigma((\varphi \circ \nu)(b/\alpha) \cdot \alpha) = 0,
\]
что представляет собой то же выражение, как когда \(\alpha \neq 0\), за исключением того, что вход для \(\kappa\) изменяется с \(\nu(b/\alpha)\) на \(\tau(b)\). В результате можно оценить эту функцию в два этапа:

\begin{enumerate}
  \item Пусть \((\alpha, b) = \mathrm{Split}_{\beta}(x)\). Если \(\alpha = 0\), пусть \(\ell = \tau(b)\); иначе пусть \(\ell = \nu(b/\alpha)\).
  \item Вернуть \(\kappa(\ell) \oplus \sigma(\varphi(\ell) \cdot \alpha)\).
\end{enumerate}

Этот процесс описан в Алгоритме~\ref{alg:TU_decomposition}, что завершает доказательство.
\end{proof}

Существование TU-разложения Бирюкова и др. является очевидным прямым следствием этой теоремы. Связь между этим разложением и оригинальной структурой TKlog дополнительно позволяет нам лучше понять некоторые паттерны, существующие в этом разложении.

Во-первых, компонент, который Бирюков и др. назвали \(\nu_0\), играет роль \(\tau\) в теореме~\ref{thm:TU_decomposition}. Так как \(\tau\) в основном является логарифмом по базе \(\alpha^{-(2m+1)}\), неудивительно, что \(\nu_0\) был замечен в \cite{PU16} как аффинно-эквивалентный логарифму.

Во-вторых, линейные слои \(\alpha\) и \(\omega\) оригинального разложения в основном разделяют \(\mathbb{GF}(2^{2m})\) на \(\mathbb{GF}(2^m)^2\), а затем соединяют его обратно, что объясняет связь, представленную в уравнении~\ref{eq:relation}. Более того, выходной слой \(\omega\) является более сложным, чем входной слой \(\alpha\), так как он также оценивает линейную часть \(\kappa\).

\subsection{Свойства, объясняемые TU-разложением}

Эффективная аппаратная реализация. Знание разложения по теореме 2 позволило Бирюкову и др. значительно улучшить реализацию \(\pi\): они сделали схему, реализующую такую перестановку, с площадью и задержкой, уменьшенными в 2.5 и 8 раз соответственно \cite{BPU16b, Table 8}. Мы можем ожидать аналогичных улучшений для всех 8-битных TKлогов.

Визуальный артефакт в LAT. Бирюков и др. использовали определенный визуальный артефакт в LAT для \(\pi\), чтобы выполнить первый шаг своего TU-разложения, а именно наличие столбцов с меньшим числом различных коэффициентов, чем другие. Они показали, что этот шаблон является прямым следствием структуры, которую они обнаружили в этом S-блоке, а именно той, что накладывается теоремой 2. Как следствие, мы можем обобщить их результат для всех экземпляров TKлогов.

\begin{lemma}
Рассмотрим экземпляр TKлога \(T_{\kappa,\alpha,s}\), пусть \(\text{Split}_{\beta}, \tau, \sigma, \nu\) и \(\phi\) будут подфункциями, используемыми для реализации его TU-разложения, как представлено в алгоритме 2, и пусть \(\phi_{\kappa} : \mathbb{GF}(2^{2m}) \to \mathbb{GF}(2^m) \times \mathbb{GF}(2^m)\) будет аффинной биекцией такой, что \(\phi_{\kappa}(\kappa(x) \oplus v) = (x, v)\). Мы дальше пусть \(\mathcal{F}\) будет перестановкой на \(\mathbb{GF} (2^m)^2\), определенной как \(\mathcal{F} = \phi_{\kappa} \circ T_{\kappa,s} \circ \text{Split}^{-1}_{\beta}\), так что она состоит из TU-разложения \(T_{\kappa,s}\) без входных и выходных линейных слоев. Тогда, для \((a_L, a_R) \neq (0, 0)\) и \(b_L \neq 0\), мы имеем
\[
\mathcal{W}_{\mathcal{F}}[(a_L, a_R), (b_L, 0)] = \left(\mathcal{W}_{\tau}[a_L, b_L] \pm 2^m\right)[a_L \neq 0],
\]
где \([a_L \neq 0] = 0\), если \(a_L = 0\), и \(1\) в противном случае.
\end{lemma}

Доказательство этой леммы по сути аналогично тому, которое дали Бирюков и др. в расширенной версии своей статьи \cite{BPU16b}. Для полноты изложения мы также приводим его в Приложении 11. 

Так как \(\tau\) по сути представляет собой дискретный логарифм, его нелинейность всегда высока, из-за чего паттерн всегда виден. Действительно, из-за этой высокой нелинейности, \(|\mathcal{W}_{\tau}[a_L, b_L]|\) может принимать значения, значительно меньшие, чем \(2^m\), что подразумевает, что низкий контраст будет всегда присутствовать.

Как отмечено в \cite{BPU16a}, авторы оригинального разложения отметили, что им повезло, поскольку стечение обстоятельств, приведшее к успеху их разложения, казалось маловероятным. По их собственным словам (страница 19 из \cite{BPU16b}):
\begin{quote}
<<[Визуальный паттерн] вызван совокупностью трех элементов:
\begin{itemize}
\item использование мультиплексора,
\item использование инверсии в конечном поле и
\item факт, что \(\nu_0\) обладает хорошей нелинейностью>>
\end{itemize}
Ирония в том, что единственная <<неудивительная>> подкомпонента \(\pi\), а именно функция обратного значения, стала одной из причин, почему нам удалось обратное инжинирить именно этот S-блок изначально. Будь [инверсия] заменена на другой (и, возможно, более слабый!) S-блок, не было бы никаких линий в LAT, которые начали бы наш процесс обратной инженерии.
\end{quote}

Теперь мы можем видеть, что, вместо удачи, все эти события являются прямым следствием структуры TKлог \(\pi\).

Случайно, это обеспечивает отличный метод для выявления такой структуры, если кто-то другой решит использовать одну из них и попытаться скрыть этот факт. В \cite{PU16} Перрин и Удовенко предложили графически изобразить дисперсию модуля абсолютной величины коэффициентов в каждой строке/столбце LAT. Они заметили, что в случае \(\pi\) было резкое падение для некоторых столбцов. Из-за леммы 3 мы видим, что этот паттерн является неотъемлемым свойством TKлог, что, следовательно, всегда будет предавать наличие такой структуры. Более того, если экземпляр TKлог будет зашифрован путем его составления с аффинными слоями, этот паттерн останется и, на самом деле, предоставит некоторую информацию о линейной части указанных аффинных слоев.

Более того, поскольку у TKлог всегда есть TU-разложение, у него всегда будет векторное пространство размерности \(n = 2^m\) внутри набора координат нулей его LAT \cite{CP19}. Этот другой паттерн также можно обнаружить и таким образом обозначить, что экземпляр является TKлогом.

CCZ-эквивалентность. Из-за теоремы 2 мы знаем, что TKлог всегда является аффинно-эквивалентной перестановке \(P : (\mathbb{GF}(2^m))^2 \to (\mathbb{F}_2^m)^2\), такой что
\[
P(b, a) = \left(T_a(b), U_{T_a(b)}(a)\right), \text{ где }\left\{\begin{array}{l}
T_a(b) = \tau(b) \cdot [a = 0] \oplus \nu(b/a) \cdot [a \neq 0], \\
U_k(a) = \sigma(a/\phi(k)),
\end{array}\right.
\]
где \([a = 0]\) — это булева функция, сопоставляющая \(0\) с \(1\) и \(a \neq 0\) с \(0\), и где \([a \neq 0] = 1 \oplus [a = 0]\).

Как объяснено в \cite{CP19}, существование такого разложения, где \(T_r\) является перестановкой для всех \(r \in \mathbb{F}_2^m\), эквивалентно возможности так называемого \(m\)-твиста, операции, которая сохраняет класс CCZ-эквивалентности, но априори не сохраняет класс AE-эквивалентности. Поскольку функциональная инверсия также сохраняет класс CCZ-эквивалентности, мы выводим следующий королларий из теоремы 2.
\begin{corollary}
Пусть \(T\) и \(U\) определены выше. И соответствующий экземпляр TKлог и его инверсия являются CCZ-эквивалентными функции \(F : \mathbb{F}_2^m \times \mathbb{GF}(2^m) \to \mathbb{F}_2^m \times \mathbb{GF}(2^m)\), такой что
\[
F(b, a) = \left(T_a^{-1}(b), U_b(a)\right), \text{ где }\left\{\begin{array}{l}
T_a^{-1}(b) = \tau^{-1}(b) \cdot [a = 0] \oplus \left(\nu^{-1}(b) \odot a\right) \cdot [a \neq 0], \\
U_b(a) = \sigma(a/\phi(b)),
\end{array}\right.
\]
\end{corollary}
\begin{proof}
По определению (см. \cite{CP19}), \(m\)-твист отображает функцию \(P : (r, \ell) \mapsto \left(T_r(\ell), U_{T_r(\ell)}(r)\right)\) в \(F : (r, \ell) \mapsto \left(T_r^{-1}(\ell), U_{\ell}(r)\right)\). Королларий следует напрямую.
\end{proof}

В случае \(\pi\) мы сгенерировали функцию \(F_\pi\), которая является CCZ-эквивалентной ей, как указано в королларии 3. Ее таблица соответствий представлена в Дополнительных материалах для полноты изложения (см. Таблицу 3). Конечно, у нее те же дифференциальный и расширенный спектры Уолша, что и у \(\pi\), и, как и у \(\pi\), все ее координаты имеют степень 7. Однако это не перестановка: у 15 элементов в её образе есть по 3 предобраза, у 75 — по 2, а у оставшихся 61 — по 1.