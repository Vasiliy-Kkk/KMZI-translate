\section{Новая информация о российских примитивах}

В этом разделе мы обсуждаем последствия того факта, что \(\pi\) является (с точностью до сдвига)
экземпляром TKlog для примитивов, использующих его. Сначала мы утверждаем, что наличие
структуры TKlog должно быть сознательным выбором от дизайнеров (раздел 5.1). Затем мы
в разделе 5.2 вводим новое представление бинарной матрицы, используемой в Стрибоге, с
помощью которого рассматриваем некоторые взаимодействия между разбиениями, сохраненными
\(\pi\), и этой линейной компонентой. Наконец, мы обсуждаем наши выводы и их последствия в разделе 5.3.

\subsection{Вероятный процесс разработки \(\pi\)}

В свете наших результатов мы можем вывести некоторую информацию о процессе дизайна
\(\pi\). Во-первых, мы устанавливаем, что число экземпляров TKlog чрезвычайно мало, что
означает, что выбор этой структуры должен быть преднамеренным. Затем, используя некоторые
экспериментальные результаты, мы получаем процесс проектирования, который дает результаты,
очень похожие на \(\pi\).

\textbf{Плотность множества TKlog.} Помимо алгоритма высокого уровня, TKlog, работающий
на 8 битах, полностью определяется тремя компонентами: примитивным полиномом \(p\)
степени 8 (существует 16 возможных вариантов), аффинной функцией \(\kappa : x \mapsto
\Lambda(x) \oplus \kappa(0)\), где \(8 \times 4\) бинарная матрица \(\Lambda\) такова, что
\(\Lambda(\mathbb{F}_{2}^{4})\) и \(\text{GF}(2^4)\) охватывают \(\text{GF}(2^8)\), и
перестановкой \(s\) на \(\mathbb{Z}/15\mathbb{Z}\). Матрица \(\Lambda\) должна быть такова,
что ее первый столбец \(\Lambda_0\) не находится в \(\text{GF}(2^4)\), второй — не входит
в \(\text{GF}(2^4) \cup (\Lambda_0 \oplus \text{GF}(2^4))\) и т. д., так что существуют
\((2^8 - 2^4)(2^8 - 2^5)(2^8 - 2^6)(2^8 - 2^7) \approx 230.3\) вариантов для этой матрицы.
Таким образом, существует около

\[
16 \hspace{10pt} \underbrace{p}_{\text{полином}} \times 230.3 \hspace{10pt} \underbrace{\Lambda}_\text{матрица} \times 2^8 \hspace{10pt} \underbrace{\kappa(0)}_\text{аффинн. функция} \times 15! \hspace{10pt} \underbrace{s}_\text{перестановка} \approx 2^{82.6}
\]

различных экземпляров TKlog на 8 битах.

Это число очень мало. Для сравнения, существует всего \(2^8! \approx 2^{1684}\) перестановок
\(\mathbb{F}_2^8\), из которых \(2^8 \times \prod_{i=0}^{7}(2^8 - 2^i) \approx 2^{70.2}\) являются
аффинными. Таким образом, число экземпляров TKlog примерно в 4000 раз больше числа аффинных
перестановок. Наша цель в этих оценках — дать представление о том, насколько мало количество
TKlogs. Генератор случайных перестановок, возвращающий аффинную перестановку, предположительно
будет намеренно генерировать такой объект. Точно так же можно предположить, что процесс
генерации, который привел дизайнеров Стрибога к выбору \(\pi\), преднамеренно вернул экземпляр TKlog.

\textbf{Утверждение.} Учитывая, как мало количество TKlog, мы уверены, что дизайнеры \(\pi\)
сознательно выбрали эту структуру.

\textbf{Экспериментальные результаты.} Насколько хороши дифференциальные и линейные
свойства экземпляров TKlog по сравнению с ожидаемыми от случайной перестановки? Чтобы
ответить на этот вопрос, мы основываемся на анализе S-блока Skipjack в \cite{BP15}, чтобы
ввести следующие понятия.

\textbf{Определение 2 (Аномалия S-блока).} Пусть \(F : \mathbb{F}_2^n \to \mathbb{F}_2^n\)
— перестановка, \(u(F)\) — ее дифференциальная однородность, а \(N_k(F)\) — число
вхождений \(k\) в ее DDT. Дифференциальная аномалия \(F\) равна

\[
A_d^F = -\log_2(\mathrm{Pr}[u(G) \leq u(F) \text{ и } N_{u(F)}(G) \leq N_{u(F)}(F)]),
\]

где вероятность берется по всем перестановкам \(G\). Если \(\ell(F)\) — линейность \(F\) и
\(N'_k(F)\) — сумма числа вхождений \(k\) и \(-k\) в LAT \(F\), то линейная аномалия \(F\)
равна

\[
A_\ell^F = -\log_2(\mathrm{Pr}[\ell(G) \leq \ell(F) \text{ и } N'_{\ell(F)}(G) \leq N'_{\ell(F)}(F)]),
\]

где вероятность берется по всем перестановкам \(G\).

S-блок с дифференциальной аномалией, близкой к 0, имеет дифференциальные свойства,
близкие к случайному S-блоку или хуже. Дифференциальная аномалия ведет себя так, как
мы ожидаем: при снижении дифференциальной однородности ниже ожидаемого значение аномалии
возрастает. Поскольку она содержит больше информации, чем дифференциальная однородность,
она позволяет сравнивать S-блоки, для которых это значение одинаково. С криптографической
точки зрения, чем выше аномалия, тем лучше, так как это означает, что S-блок обеспечит
лучшую защиту от дифференциальных атак. То же самое можно сказать о линейной аномалии.

В \cite{BP15} Бирюков и Перрин предоставили формулы для вычисления дифференциальных и
линейных аномалий, основанные на статистическом распределении коэффициентов DDT и LAT,
представленных в \cite{DR07}. Они также показали, что линейная аномалия S-блока Skipjack
равна 55.4, поэтому этот компонент не мог быть сгенерирован случайно.

С целью получения дополнительной информации о процессе проектирования \(\pi\), мы
сгенерировали \(10^6\) случайных 8-битных экземпляров TKlog. Мы изобразили дифференциальные
и линейные аномалии каждого из них на двумерной диаграмме, представленной на рисунке 3.
Каждому экземпляру соответствует светло-серый пункт; более темные точки возникают, когда
несколько экземпляров имеют одни и те же дифференциальные и линейные аномалии. Мы также
включили аномалии \(\pi\), \(\log_{\text{FLY}} \alpha\) и \(\log_{\text{HN}} \alpha\) в ту же
диаграмму для сравнения.

Как видно, дифференциальные и линейные аномалии \(\pi\) несколько хороши, но не
исключительны по сравнению с теми, что ожидаются от случайного TKlog. Более конкретно,
8-битный экземпляр TKlog имеет как минимум такие же дифференциальные и линейные аномалии,
как и у \(\pi\), с вероятностью около \(2^{-10.6}\), и несложно получить гораздо лучшие
экземпляры. Они также ниже, чем у \(\log_{\text{FLY}} \alpha\) и \(\log_{\text{HN}} \alpha\).

Тем не менее, ни один из наших случайных экземпляров не имеет лучшей дифференциальной
однородности или линейности, чем \(\pi\) (включая \(\log_{\text{FLY}} \alpha\) и
\(\log_{\text{HN}} \alpha\)). Более того, \(\pi\) находится в области на рисунке 3,
содержащей большинство экземпляров с той же дифференциальной однородностью и линейностью.
Таким образом, его аномалии соответствуют таковым у случайного экземпляра TKlog с той
же дифференциальной однородностью и линейностью.

\textbf{Очерк процесса проектирования.} В свете вышеприведенных экспериментальных
результатов, мы можем увидеть, что следующий процесс проектирования приведет к
результату очень похожему на \(\pi\).

\begin{enumerate}
  \item Определите, что наилучшая возможная дифференциальная однородность для 8-битного
  экземпляра TKlog — это 8, а наилучшая линейность — 56, например, с помощью
  обширных компьютерных симуляций.
  
  \item Выберите случайно экземпляр TKlog среди тех, у которых имеются указанные
  дифференциальная однородность и линейность, — не принимая во внимание аномалию.
\end{enumerate}

Эта стратегия естественна до тех пор, пока существует причина навязывать использование
TKlog — хотя мы не можем придумать ни одной. Поскольку как дифференциальные, так и
линейные аномалии \(\pi\) ниже, чем у \(\log_{\text{FLY}} \alpha\) и \(\log_{\text{HN}} \alpha\),
цель использования TKlog в этом случае не может заключаться в улучшении
криптографических свойств дискретного логарифма. Более важно, что крайне
сильные алгебраические свойства таких компонентов, которые мы описали в
разделе 3.2, априори требуют осторожности; еще больше в случае Стрибога.
Действительно, как мы объясняем ниже, его линейный слой нетривиально
взаимодействует с соответствующими разбиениями.

\section{Линейное отображение в Streebog}

Бинарная матрица, соответствующая L-операции Streebog, представлена на \autoref{fig:matrix}, где черный пиксель соответствует 1, а белый — 0.

Как видно, матрица обладает значительной структурой. В \cite{KK13} Казимиров и Казимирова показали, что она представима как композиция:

\begin{itemize}
    \item слой 8-битных линейных перестановок \(\ell\), который просто инвертирует порядок битов в каждом байте,
    \item умножение на $8 x 8$ MDS-матрицу $GF(2^8) = F_2[X]/P_{KK}(X)$, где \(P_{KK}(X) = X^8 \oplus X^6 \oplus X^5 \oplus X^4 \oplus 1\) — примитивный многочлен степени 8,
    \item обратное \(\ell\).
\end{itemize}

Мы использовали очень прямой подход для упрощения этой структуры: задавали каждый байт в строке равным 1 по очереди, умножали его на L, затем записывали байты как элементы $GF(2^8) = F_2[X]/p_{\text{min}}(X)$, генерируя матрицу LF такую, что:

Многочлен, используемый Казимировым и Казимировой, является обратным для \(p_{\text{min}}\), т.е. \(P_{KK}(1/X) = p_{\text{min}}(X)/X^8\). Задним числом было очевидно, что обращение порядка бит в байте в их выражении линейного слоя может быть удалено, используя этот многочлен.

В итоге, если мы обозначим $A$ как $8 x 8$ матрицу элементов $GF(2^8)$, соответствующую внутреннему состоянию Streebog'а, и обозначим P транспонирование $A$ (как указано в спецификации Streebog), применение всей линейной части раундовой функции Streebog можно записать как
\[
(L \circ P)(A) = A^T \times LF,
\]
где “$x$” обозначает обычное умножение матриц.

\subsection{Аддитивные и мультипликативные классы смежности}

Подполе $GF(2^4)^*$ имеет особую связь с $L$. Действительно, применение умножения матрицы Streebog к вектору \(x_i = [0, \ldots, 0, x, 0, \ldots, 0]\) из $GF(2^8)^8$, таким, что \(x_i^i = x\) и \(x_i^k = 0\) при \(k \neq i\), эквивалентно вычислению
\[
v = x_i \times LF = [LF_{i,0} \cdot x, ..., LF_{i,7} \cdot x],
\]
так что если \(x \in GF(2^m)^*\), то \(v_j \in LF_{i,j} \cdot GF(2^m)^*\), т.е., оно отображает подполе на его мультипликативные классы смежности. Однако, непонятно, что происходит, когда активны несколько ячеек входного вектора.

\subsection{Линейное отображение в Kuznyechik}
Линейный слой Kuznyechik определяется как РСЛО с 16 ячейками, каждая из которых является элементом $GF(2^8)$, и который сдвигается на 16 шагов. Его также можно представить как умножение на $16 x 16$ матрицу. Однако представление элементов поля использует другой многочлен, а именно \(p_{\text{kuz}}(X) = X^8 \oplus X^7 \oplus X^6 \oplus X \oplus 1\). В то время как \(p_{\text{min}}(X) = X^8 \oplus X^4 \oplus X^3 \oplus X^2 \oplus 1\) — первый примитивный многочлен степени 8 в лексикографическом порядке, \(p_{\text{kuz}}\) — последний такой многочлен веса 5, как указано в \cite{LN97}.

В отличие от умножения на матрицу в Streebog, в Kuznyechik его нельзя записать как умножение в $F_2[X]/p_{\text{min}}(X)$, так что распространение классов смежности, описанное выше для хеш-функции, не применимо для блочного шифра.

\subsection{Обсуждение}

Последствия сохранения разделения на классы смежности п и его нетривиального взаимодействия с линейным слоем оценить сложно.

В литературе есть другие S-блоки, отображающие классы смежности на классы смежности. Например, мономиалы отображают мультипликативные классы смежности подполе на мультипликативные классы смежности подполе. Если \(F: x \mapsto x^d\) — перестановка $GF(2^{2m})$, то
\[
F(\alpha_i \cdot GF(2^m)) = \alpha^{d \cdot i} \cdot GF(2^m).
\]
Если убрать их аффинные компоненты, S-блоки AES \cite{AES01} и Misty1 \cite{Mat97} (среди многих других) демонстрируют это поведение. Однако, несмотря на их присутствие в некоторых очень заметных целях, мультипликативные классы смежности никогда не использовались в симметричной криптоанализе. Заметим, что дизайнеры алгоритмов всегда сочетают обратное с несвязанными аффинными слоями, чтобы разрушить его алгебраическую структуру. Такое консервативное решение, вероятно, предназначено для предотвращения использования мультипликативных классов смежности для атаки на эти шифры на практике.

Ситуация другая с аддитивными классами смежности. В \cite{BBF16} авторы намеренно построили S-блок, отображающий аддитивные классы смежности на аддитивные классы смежности, с явной целью использования этого шаблона в качестве «черного хода». Они показывают, что такое разделение может сохраняться, если линейный слой выбран с умом и может сохраняться для произвольного количества раундов. Причина, по которой Баннер и другие рассматривали аддитивные классы смежности, заключается в следующем наблюдении.

% \begin{remark}
% Если \(F_k: x \mapsto x \oplus k\) — добавление ключа в $F^n_2$, и \(V\) — векторное подпространство $F^n_2$, тогда \(F_k(c \oplus V) = (k \oplus c) \oplus V\), так что разделение $F^n_2$ на аддитивные классы смежности \(V\) сохраняется под действием \(F_k\) независимо от \(k\).
% \end{remark}

Таким образом, в шифре Баннера и других, расписание ключей может быть произвольно сложным, не мешая наличию 'черного хода'. Это свойство не разделяет разделение на мультипликативные классы смежности.

Кроме TKlog мы можем вспомнить лишь один случай, когда классы смежности отображаются на другую разделение, а именно дискретные логарифмы. Действительно, логарифм типа Хакала-Ньюберг, действующий на $GF(2^{2m})$, который отображает \(\alpha^{2^m-1}\) в $0$, всегда отображает мультипликативные классы смежности подполе на аддитивные классы смежности $Z/(2^m - 1)Z$. В этом случае умножение производится в конечном поле, а сложение над целыми числами. Поскольку эти две операции совершенно разные, маловероятно, что эта характеристика помогает в криптоанализе.

В конце концов, рассматривая влияние классов смежности на симметричные примитивы, мы имеем одну из следующих ситуаций:
\begin{enumerate}
    \item разделение на классы смежности не может быть итеративным, поскольку входное и выходное разделения находятся в совершенно разных структурах (случай логарифма);
    \item хотя S-блок и линейный слой определены над схожими структурами, была добавлена небольшая функция с явной целью разрушения этого сходства (случай AES и аффинной перестановки, использованной в его S-блоке);
    \item S-блок и линейный слой были выбраны с выравненными структурами, которые сохраняют ту же самую разделение, чтобы намеренно ввести 'черный ход' в блочном шифре (защищенный шифр \cite{BBF16}).
\end{enumerate}

По-видимому, Kuznyechik относится ко второму случаю. Хотя разработчики не раскрыли своего анализа безопасности, было бы логично, чтобы они выбрали многочлен, используемый для определения конечного поля, в котором функционирует линейный слой, чтобы не 'выравнивать' его с структурой, используемой для построения \(\pi\).

Однако Streebog не подпадает ни в одну из этих категорий. Входные и выходные классы смежности определены для той же структуры (конечного поля), так что он не в первой ситуации. S-блок мог бы быть комбинирован с аффинным слоем, разрушающим его отношение с $GF(2^8)$ (как в AES), или линейный слой мог бы быть определен для другого конечного поля (как в Kuznyechik), но ни то, ни другое не происходит, так что он не подпадает и под вторую категорию. Однако, хотя линейный слой определен над той же структурой, что и классы смежности, сохраняемые S-блоком, эти разделения отличаются, и непонятно, как они могут взаимодействовать с умножением матрицы. Поэтому не очевидно, что Streebog относится к третьей категории, и следующий вопрос остается открытым.

% \begin{problem}
% Существует ли способ использовать сохранение разделения \(\pi\) для атаки на Streebog?
% \end{problem}