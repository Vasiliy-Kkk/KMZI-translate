\section{TKlog}

В этом разделе мы вводим новый тип перестановки, который мы называем TKlog, особым случаем которого является \(\pi\). Поскольку Стрибог и Кузнечик были разработаны "ТК-26"\footnote{[Тут сноска о значении ТК-26]}, мы используем буквы "TK" для именования этой структуры. Она обладает логарифмическими свойствами, хотя и отображает \(\mathrm{GF}(2^{2m})\) на себя, а не на \(\mathbb{Z}/2^{2m}\mathbb{Z}\), отсюда и "log" часть имени. Более точно, она отображает разбиение \(\mathrm{GF}(2^{2m})\) на мультипликативные классы \(\mathrm{GF}(2^m)^*\) в её разбиение на аддитивные классы \(\mathrm{GF}(2^m)\), и её ограничение на каждый мультипликативный класс по существу одинаково для всех классов. Мы определяем эту структуру в секции 3.1 и представляем детали этого сохраняющего разбиение свойства в секции 3.2.

\subsection{TKlog Перестановочная Структура}

\textbf{TKlog.} Пусть \(\mathrm{GF}(2^{2m}) = \mathbb{F}_2[X]/p(X)\) — конечное поле чётной степени, определённое примитивным многочленом \(p\). Мультипликативная подгруппа \(\mathrm{GF}(2^{2m})^*\) циклична и генерируется \(\alpha\), такой что \(p(\alpha) = 0\). В этом контексте \(\alpha^{2m+1}\) является генератором мультипликативной подгруппы подполей \(\mathrm{GF}(2^m)\).

\textbf{Определение 1 (TKlog).} TKlog — это перестановка, действующая на \(\mathrm{GF}(2^{2m}) = \mathbb{F}_2[X]/p(X)\) для некоторого примитивного многочлена \(p\) с корнем \(\alpha\). Она параметризована:
\begin{itemize}
    \item аффинной функцией \(\kappa : \mathbb{F}^m_2 \to \mathrm{GF}(2^{2m})\) такой, что любое \(x \in \mathrm{GF}(2^{2m})\) может быть записано как \(x = x_m \oplus \kappa(x_\kappa) \oplus \kappa(0)\) для некоторых \(x_m \in \mathrm{GF}(2^m)\) и \(x_\kappa \in \mathbb{F}^m_2\),\footnote{[Дополнительное пояснение может быть добавлено здесь при необходимости]}
    \item перестановкой \(s\) из \(\mathbb{Z}/(2^m - 1)\mathbb{Z}\).
\end{itemize}
Соответствующий TKlog обозначается \(T_{\kappa,s}\) и действует следующим образом:
% \[
% \begin{cases}
% T_{\kappa,s}(0) = \kappa(0), \
% T_{\kappa,s}((\alpha^{2m+1})^j) = \kappa(2^m - j), & \text{для } 1 \leq j \leq 2^m - 1, \
% T_{\kappa,s}(\alpha^i + (2m+1)j) = \kappa(2^m - i) \oplus (\alpha^{2m+1})^{s(j)}, & \text{для } 0 < i, 0 \leq j < 2^m - 1,
% \end{cases}
% \]
где факт, что \(1 \leq j \leq 2^m - 1\) вместо \(0 \leq j < 2^m - 1\), когда \(x \in \mathrm{GF}(2^m)^*\), происходит из неявного использования логарифма Фэнга и др. Алгоритм 3 (в приложении A9) оценивает такую перестановку.