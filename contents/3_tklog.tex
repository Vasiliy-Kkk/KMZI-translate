\section{TKlog}

В этом разделе вводится новый тип перестановки TKlog, частным случаем которого является \(\pi\). Стрибог и Кузнечик были разработаны "ТК-26"\footnote{Официальное название этой организации "Технический Комитет По Стандартизации «Криптографическая Защита Информации»"}, поэтому используется "TK" в TKlog. Она обладает логарифмическими свойствами, хотя и отображает \(\mathrm{GF}(2^{2m})\) в себя, а не в \(\mathbb{Z}/2^{2m}\mathbb{Z}\), отсюда и "log" в TKlog. Более точно, она отображает разложение \(\mathrm{GF}(2^{2m})\) на мультипликативные смежные классы \(\mathrm{GF}(2^m)^*\) в разбиение на аддитивные смежные классы \(\mathrm{GF}(2^m)\), и её ограничение на каждый мультипликативный смежный класс класс по сути одинаково для всех смежных классов. Эта структура будет определена в разделе 3.1, а в разделе 3.2 будут представлены детали этого свойства, сохраняющего разбиение.

\subsection{Перестановочная структура TKlog}

\textbf{TKlog.} Пусть \(\mathrm{GF}(2^{2m}) = \mathbb{F}_2[X]/p(X)\) — конечное поле, заданное примитивным многочленом \(p\). Мультипликативная подгруппа \(\mathrm{GF}(2^{2m})^*\) циклична и задается порождающим элементом \(\alpha\), таким что \(p(\alpha) = 0\). В этом контексте \(\alpha^{2m+1}\) является порождающим элементом мультипликативной подгруппы подполя \(\mathrm{GF}(2^m)\).

\textbf{Определение 1} \((TKlog)\). TKlog — это перестановка, действующая над \(\mathrm{GF}(2^{2m}) = \mathbb{F}_2[X]/p(X)\) для некоторого примитивного многочлена \(p\) с корнем \(\alpha\). Она задается:
\begin{itemize}
    \item аффинной функцией \(\kappa : \mathbb{F}^m_2 \to \mathrm{GF}(2^{2m})\) такой, что любой \(x \in \mathrm{GF}(2^{2m})\) может быть записан как \(x = x_m \oplus \kappa(x_\kappa) \oplus \kappa(0)\) для некоторых \(x_m \in \mathrm{GF}(2^m)\) и \(x_\kappa \in \mathbb{F}^m_2\)\footnote{Эквивалентно, можно записать, что линейная часть отображения \(\kappa\), которая отображает \(y \in \mathbb{F}_2^m\) в \(\kappa(y) \oplus \kappa(0) \in \mathbb{GF}(2^{2m})\), отображает \(\mathbb{F}_2^m\) в векторное пространство в прямой сумме с подполем \(\mathbb{GF}(2^m)\).};
    \item перестановкой \(s\) из \(\mathbb{Z}/(2^m - 1)\mathbb{Z}\).
\end{itemize}
Соответствующий TKlog обозначается \(\mathscr{T}_{\kappa,s}\) и действует следующим образом:

$$
\begin{cases}
  \mathscr{T}_{\kappa, s}(0) & =\kappa(0), \\
  \mathscr{T}_{\kappa, s}\left(\left(\alpha^{2^m+1}\right)^j\right) & =\kappa\left(2^m-j\right), \text { для } 1 \leq j \leq 2^m-1, \\
  \mathscr{T}_{\kappa, s}\left(\alpha^{i+\left(2^m+1\right) j}\right) & =\kappa\left(2^m-i\right) \oplus\left(\alpha^{2^m+1}\right)^{s(j)}, \text { для } 0<i, 0 \leq j<2^m-1,
\end{cases}
$$ где тот факт, что \(1 \leq j \leq 2^m - 1\), а не \(0 \leq j < 2^m - 1\), когда \(x \in \mathrm{GF}(2^m)^*\), следует из неявного использования логарифма Фэнга и др. Алгоритм 3 (в Разделе 9) оценивает такую перестановку.

Чтобы убедиться в том, что пример TKlog, определённый выше, действительно является перестановкой, следует определить его функциональную обратную функцию, TKexp. Она использует \(\Phi_\kappa: \text{GF}(2^{2m}) \rightarrow \mathbb{F}^m_2 \times \text{GF}(2^m)\), которая является аффинной перестановкой такой, что:

\[
\Phi_\kappa(\kappa(x) \oplus v) = (x, v)
\] для \((x, v) \in \mathbb{F}^m_2 \times \text{GF}(2^m)\). TKexp работает следующим образом:
$$
\left\{\begin{array}{l}
  \left(\mathscr{T}_{\kappa, s}^{-1} \circ \Phi_\kappa^{-1}\right)(0,0)=0, \\
  \left(\mathscr{T}_{\kappa, s}^{-1} \circ \Phi_\kappa^{-1}\right)(k, 0)=\alpha^{\left(2^m+1\right)\left(2^m-k\right)}, \\
  \left(\mathscr{T}_{\kappa, s}^{-1} \circ \Phi_\kappa^{-1}\right)(k, v)=\alpha^{2^m-k+\left(2^m+1\right) s^{-1}(j)} \text { где } j=\log _\alpha^{F L Y}(v) /\left(2^m+1\right) .
  \end{array}\right.
$$

Альтернативно, это может быть вычислено с использованием Алгоритма 4 (в Разделе 9).

\textbf{О вычитании} \((Substraction)\) Целочисленное вычитание на входе \(\kappa\) необходимо из-за того, что \(i \in \{1, \ldots, 2^m\}\), поэтому двоичное представление \(i\) не всегда помещается в \(m\) бит. Cледовательно, нужна небольшая функция, которая отображает \(\{1, \ldots, 2^m\}\) в \(\{0, \ldots, 2^{m - 1}\}\), поскольку случай \(i = 0\) обрабатывается отдельно. Естественным выбором, очевидно, было бы \(i \mapsto i - 1\), но в этом случае потребуется другая функция для случая, когда \(i = 0\). Действительно, поскольку используется логарифм FLY, получается \(j \in \{1, \ldots, 2^{m - 1}\}\) для \(i = 0\). Таким образом, нужно будет получать \(\kappa\) из двух различных функций в зависимости от того, будет ли \(i = 0\). С другой стороны, функция \(i \leftarrow 2^m - i\) отображает и \(\{1, \ldots, 2^m\}\) в \(\{0, \ldots, 2^m - 1\}\) и \(\{1, \ldots, 2^m - 1\}\) в себя, то есть она может использоваться в обоих случаях.

Нам не удалось найти такую простую функцию, когда используется логарифм \(\log_{\alpha}^{\text{HN}}\).

\textbf{Частный случай} $\bm{\pi}$. S-блок \(\pi\) является TKlog, работающим с 8 битами. Для целей реализации будем отождествлять \(\text{GF}(2^8)\) с \(\mathbb{F}^8_2\), используя метод, описанный в Разделе 2.1. Когда \(\pi\) записывается как пример TKlog, она использует следующие компоненты:

\begin{itemize}
    \item конечное поле \(\text{GF}(2^8) = \mathbb{F}_2[X]/p_{\text{min}}(X)\), где \(p_{\text{min}}(X) = X^8 \oplus X^4 \oplus X^3 \oplus X^2 \oplus 1\) и его корень \(\alpha\);
    \item аффинная функция \(\kappa\), отображающая \(\mathbb{F}^4_2\) в \(\mathbb{F}^8_2\) так, что \(\kappa(0) = 0\)x\(\text{FC}\) с линейной частью \(\Lambda\), определённой как
    \[
    \Lambda(1) = 12, \quad \Lambda(2) = 26, \quad \Lambda(4) = 24, \quad \Lambda(8) = 30,
    \] где числа записаны в шестнадцатеричном формате и где линейная функция \(\Lambda\) проверяет \(\{x_4 \oplus \Lambda(y), x_4 \in \text{GF}(2^4), y \in \mathbb{F}^4_2\} = \text{GF}(2^8)\);
    \item перестановка \(s\) из \(\mathbb{Z}/15\mathbb{Z}\), определённая в Таблице \eqref{tab:tab1}.
\end{itemize}

\begin{table}    
  \caption{Таблица поиска перестановки \(s\) из \(\mathbb{Z}/15\mathbb{Z}\)}
  \begin{tabular}{cccccccccccccccc}
    \hline$x$ & 0 & 1 & 2 & 3 & 4 & 5 & 6 & 7 & 8 & 9 & 10 & 11 & 12 & 13 & 14 \\
    \hline$s(x)$ & 0 & 12 & 9 & 8 & 7 & 4 & 14 & 6 & 5 & 10 & 2 & 11 & 1 & 3 & 13 \\
    \hline
  \end{tabular}
  \label{tab:tab1}
\end{table}

Оценка \(\pi\) с использованием этой структуры резюмирована в Алгоритме \eqref{alg:alg01}. Реализация \(\pi\), основанная на этом алгоритме, приведена как скрипт на SAGE \cite{Dev17} в Разделе 12. Примечательно, что функция логарифма по умолчанию в SAGE — это \(\log_{\alpha}^{\text{FLY}}\), что упрощает реализацию функции.

В итоге действительно удалось получить структуру TKlog, сначала разложив \(\pi\), а затем обобщив найденную структуру. Краткое описание процесса реверс-инжиниринга приведена в Разделе 10.

\begin{algorithm}[htp!]
    \KwData{$x \in \text{GF}(2^8)$}
    \eIf{$x = 0$} {
      return $\kappa(0)$\;
    }{
      $k = \log_{\alpha}^{\text{FLY}}(x)$\;
      $i \gets k \bmod 17$; \quad $j \gets k/17 $ \Comment*[r]{$x = \alpha^{i+17j}$}
      \eIf{$i = 0$} { 
        return $\kappa(16 - j)$ \Comment*[l]{$x \in \text{GF}(2^4)$; $i = 0 \Rightarrow j \in \{1, \ldots, 15\}$}
      }{
        return $\kappa(16 - i) \oplus (\alpha^{17})^{s(j)}$ \Comment*[r]{$i \neq 0$, поэтому $16 - i \neq 16$}
      }
    }
  \caption{Новая декомпозиция \(\pi\)}
  \label{alg:alg01}
\end{algorithm}

\subsection{Смежные классы к смежным классам}

\subsubsection{Свойство сохранения разбиения}

Напомним, что \(\mathrm{GF}(2^m)^*\) это поле размера \(2^m\) без 0 и что оно содержится в \(\mathrm{GF}(2^{2m})^*\).
Пусть \(\alpha\) является мультипликативным генератором \(\mathrm{GF}(2^{2m})^*\) так, что \(\alpha^{2^m+1}\) является мультипликативным генератором \(\mathrm{GF}(2^m)^*\). Поле \(\mathrm{GF}(2^{2m})\) можно представить двумя различными способами с использованием мультипликативных классов смежности их \(\mathrm{GF}(2^m)^*\) с одной стороны и аддитивных классов смежности с другой:

\begin{itemize}
  \item Все элементы \(\mathrm{GF}(2^{2m})^*\) можно записать как \(\alpha^{i+(2^m+1)j} = \alpha^i (\alpha^{2^m+1})^j\), так что
  \[
  \mathrm{GF}(2^{2m}) = \{0\} \cup \left(\bigcup_{i=0}^{2^m} \alpha^i \, \cdot \, \mathrm{GF}(2^m)^*\right) = \mathrm{GF}(2^m) \cup \left(\bigcup_{i=1}^{2^m} \alpha^i \, \cdot \, \mathrm{GF}(2^m)^*\right).
  \]
  \item Поскольку \(\mathrm{GF}(2^m)\) является векторным пространством размерности \(m\), существует векторное пространство \(W\) из элементов \(\mathrm{GF}(2^{2m})\) размерности \(m\) так, что \(\mathrm{GF}(2^{2m})\) является прямой суммой \(W\) и \(\mathrm{GF}(2^m)\). В этом случае мы можем записать
  \[
  \mathrm{GF}(2^{2m}) = \bigcup_{w \in W} w \oplus \mathrm{GF}(2^m) = W \cup \left(\bigcup_{w \in W} w \oplus \mathrm{GF}(2^m)^*\right).
  \]
\end{itemize}

Оба \(W\) и \(\mathrm{GF}(2^m)\) являются векторными пространствами размерности \(m\), и в каждой декомпозиции используется \(2^m\) классов смежности \(\mathrm{GF}(2^m)^*\). Таким образом, можно отобразить эти разложения друг в друга. Это как раз то, что делает TKlog. Более формально, справедливо следующее утверждение.

\textbf{Теорема 1 (Классы смежности к классам смежности).} Пусть \(T_{\kappa,s} : \mathrm{GF}(2^{2m}) \to \mathrm{GF}(2^{2m})\) является допустимым примером TKlog. Тогда всегда верны следующие равенства:
\[
\begin{cases}
T_{\kappa,s}(\mathrm{GF}(2^m)) = \kappa(\mathbb{F}_2^m) \
T_{\kappa,s}(\alpha^i \, \cdot \, \mathrm{GF}(2^m)^*) = \kappa(2^m - i) \oplus \mathrm{GF}(2^m)^*, & \forall i \neq 0.
\end{cases}
\]

\textbf{Следствие 1 (Векторные пространства к аффинным пространствам).} Пусть \(T_{\kappa,s} : \mathrm{GF}(2^{2m}) \to \mathrm{GF}(2^{2m})\) является допустимым примером TKlog. Тогда всегда верно, что
\[
T_{\kappa,s}(\mathrm{GF}(2^m)) = \kappa(\mathbb{F}_2^m) \text{ и } T_{\kappa,s}(\alpha^{2^m} \cdot \mathrm{GF}(2^m)) = \kappa(0) \oplus \mathrm{GF}(2^m),
\]

где все пространства, участвующие в этих равенствах, имеют размерность \( m \). Кроме того,
\[
GF(2^{2m}) = \{ x \oplus y, x \in GF(2^m), y \in \alpha^{2m} \odot GF(2^m) \}
= \{ x \oplus y, x \in \kappa(0) \oplus GF(2^m), y \in \kappa(0) \oplus \kappa (\mathbb{F}_{2}^{m}) \},
\]
таким образом, \( \text{TKlog} \) отображает два векторных пространства размерности \( m \), порождающих \( GF(2^{2m}) \), в два аффинных пространства размерности \( m \), порождающих \( GF(2^{2m}) \).

Поскольку \( \pi \) является экземпляром TKlog, Теорема 1 подразумевает, что она проверяет следующие равенства множеств:
\[
\pi\left( GF(2^4) \right) = \kappa(\mathbb{F}_{2}^{4})
\]
\[
\pi\left( \alpha^i \odot GF(2^4)^{\ast} \right) = \kappa(16 - i) \oplus GF(2^m)^{\ast}, \quad \forall i \neq 0,
\]
и применение Короллария 1 даёт \(\pi(\alpha^{16} \odot GF(2^4)) = \kappa(0) \oplus GF(2^4)\). Эти равенства суммированы на Рисунке 2, где взаимосвязи между полными аффинными пространствами представлены пунктирными и толстыми стрелками, тогда как взаимосвязи, соединяющие множества размера \(2^m - 1\), представлены сплошными тонкими.

TODO: вставить картинку

\subsubsection{Простота свойств TKlog}
Разбиения, рассматриваемые в Теореме 1, имеют простые алгебраические описания. Действительно,
\( x, y \in GF(2^{2m})^{\ast} \) принадлежат одному и тому же аддитивному смежному классу \( GF(2^m)^{\ast} \), если и только если \( \text{Tr}_m(x) = \text{Tr}_m(y) \).
Отметим, что
\[
\left(\alpha^{i+(2m+1)j}\right)^{2^m-1} = \alpha^{(2^m-1)i+(2^{2m}-1)j} = \alpha^{(2^m-1)i},
\]
таким образом,
\[
\left(\alpha^{i+(2m+1)j}\right)^{2^m-1} = \left(\alpha^{i'+(2m+1)j'}\right)^{2^m-1}
\]
если и только если \( i = i' \). Следовательно, \( x, y \in GF(2^{2m})^{\ast} \) принадлежат одному и тому же мультипликативному смежному классу \( GF(2^m)^{\ast} \), если и только если \( x^{2^m-1} = y^{2^m-1} \).
Кроме того, \( x \in GF(2^m)^{\ast} \) если и только если \( x^{2^m-1} = 1 \). Мы получаем следующий королларий Теоремы 1.

\paragraph{Королларий 2.} Пусть \( T_{\kappa,s} : GF(2^{2m}) \to GF(2^{2m}) \) является действительным экземпляром TKlog. Тогда мы всегда имеем
\[
x^{2^m-1} = y^{2^m-1} \neq 1 \Leftrightarrow \text{Tr}_m(T_{\kappa,s}(x)) = \text{Tr}_m(T_{\kappa,s}(y)) \neq \text{Tr}_m(\kappa(0)).
\]
Как следствие, для любой константы \( c \in GF(2^{2m}) \setminus \{0, 1\} \), мы имеем следующее следствие, включающее только линейные уравнения:
\[
x^{2^m} \oplus cx = y^{2^m} \oplus cy = 0 \implies (T_{\kappa,s}(x))^{2^m} \oplus T_{\kappa,s}(x) = (T_{\kappa,s}(y))^{2^m} \oplus T_{\kappa,s}(y).
\]
Взаимодействие TKlog с этими двумя разбиениями выходит за рамки простого отображения одного на другой. Действительно, рассмотрим более общую структуру, соответствующую перестановкам \( P \) таких, что:
\[
P :
\begin{cases}
0 \mapsto \kappa (t_0(0)) \\
\alpha^{(2m+1)j} \mapsto \kappa (t_0(2^m - j)) \\
\alpha^{i+(2m+1)j} \mapsto \kappa (t_1(2^m - i)) \oplus (\alpha^{2m+1}) s_i(j) \text{ for } i > 0,
\end{cases}
\]
где \( t_0 \) и \( t_1 \) являются перестановками \( \mathbb{F}_{2}^{m} \), и где \( s_i \) является перестановкой \( \mathbb{Z}/(2^m - 1)\mathbb{Z} \) для всех \( i \in \mathbb{Z}/(2m + 1)\mathbb{Z} \). Любая такая \( P \) является перестановкой с точной свойством сохранения разбиения, описанного в Теореме 1, но её вклад в \( GF (2^m) \) и в \( \kappa(\mathbb{F}_{2}^{m}) \) зависит как от \( i \), так и от \( j \), даже если мы ограничимся \( i > 0 \). В случае с TKlogs это не так; эти перестановки намного проще.

\textbf{Лемма 2} \textit{(Свойство разделения).} Пусть \( T_{\kappa,s} : GF(2^{2m}) \to GF(2^{2m}) \) является действительным экземпляром TKlog. Тогда для любых \( i, j \), таких что \( 0 < i \leq 2m \) и \( 0 \leq j < 2^m - 1 \), выполнено:
\[
T_{\kappa,s}(\alpha^{i+(2m+1)j}) = \kappa(2^m - i)
\quad \subset \kappa(\mathbb{F}_{2}^{m})
\oplus (\alpha^{2m+1})^s(j)
\quad \subset GF(2^m)^{\ast},
\]
так что вклад \( i \) ограничен \( \kappa(\mathbb{F}_{2}^{m}) \), а вклад \( j \) ограничен \( GF(2^m) \).
Другими словами, TKlog взаимодействует с каждым мультипликативным смежным классом, кроме \( GF(2^m)^{\ast} \), точно так же, хотя это свойство никоим образом не подразумевается Теоремой 1.

Мы не смогли найти никаких атак, использующих эти удивительные свойства \( \pi \). Однако мы обнаружили, что эти разбиения взаимодействуют нетривиально с линейным слоем Streebog. Мы обсудим последствия присутствия этой структуры в \( \pi \) в Разделе 5.