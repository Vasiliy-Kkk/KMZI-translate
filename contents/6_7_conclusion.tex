\section{Заключение}
В рамках данной работы удалось извлечь новую структуру из \(\pi\), которая, как утверждается в этой работе, была изначально задуманна ее разработчиками. Ее обобщение, TKlog, получается путем композиции дискретного логарифма с простым арифметическим слоем. TKlog объясняет оба предыдущих разложения \(\pi\), таким образом обеспечивая недостающее звено между этими двумя результатами.

Знание об этой декомпозиции позволило объяснить очень специфическое свойство сохранения разбиения \(\pi\). Удивительно, но также было найдено новое выражение линейного слоя Стрибога, выраженное в том же конечном поле что и \(\pi\). Хотя пока не получается использовать эти свойства для атаки на эту хеш-функцию, ставится под сомнение вопрос уместности этого решения. Действительно, при работе с компонентами, определенными над идентичными математическими структурами, академические разработчики нарушают это соответствие, например, составляя их S-блоки с несвязанными аффинными перестановками. Можно считать, что было бы более осторожно поступить так же в Стрибог.
