\section{Недостающее звено}
Структура TKlog является недостающим звеном между двумя предыдущими разложениями \(\pi\). Ее связь с логарифмическими разложениями \cite{PU16} естественна, так как обе состоят из дискретного логарифма с последующей арифметикой. Тот факт, что \(\pi\) имеет TU-разложение, остается удивительным. В Разделе 3.1 будет показано, что это всегда так для TKlog. В Разделе 3.2 также будут перечислены некоторые следствия этого свойства.

\subsection{TU-разложение всегда существует}
TKlog всегда можно выразить способом, очень похожим на первое разложение Бирюкова и др. \cite{BPU16a}. Для того чтобы установить это, введем следующую лемму:

\textbf{Лемма 2.} Существует функция \(\gamma : (\mathbb{Z}/(2m + 1)\mathbb{Z})^* \to \mathbb{Z}/(2m - 1)\mathbb{Z}\) такая, что 
\[
\{x \in GF(2^{2m}), \mathrm{Tr}(x) = 1\} = \{\alpha^{i + (2m+1)\gamma(i)}, i \in \mathbb{Z}/(2m + 1)Z^*\}.
\]

\begin{proof}
Пусть \((a, b) = \mathrm{Split}_{\beta}(x)\). Если \(x \notin GF(2^m)\), то \(a \neq 0\) и можно записать \(x = a(\beta \oplus c)\), где \(c = b/a\) является элементом \(GF(2^m)\), таким, что \(\mathrm{Tr}(\beta \oplus c) = \mathrm{Tr}(\beta) = 1\). Так как такое разложение существует для всех \(x \notin GF(2^m)\), получается, что множество 
\[
\{\mathrm{log^{FLY}}_{\alpha}(c \oplus \beta), c \in GF(2^m)\}
\]
должно содержать представителя каждого класса эквивалентности по модулю \(2^m + 1\), так как в противном случае некоторые элементы \(\alpha^{i + (2^m+1)j}\) при \(i \neq 0\) не могли бы быть записаны в виде \(a(\beta \oplus c)\).

Тогда \(\{\mathrm{log^{FLY}}_{\alpha}(c \oplus \beta) \mod (2^m + 1), c \in GF(2^m)\} = \{1, \ldots, 2^m\}\),
и, следовательно, должна существовать функция \(\gamma\), представленная выше, такая что:
\[
\{\mathrm{log^{FLY}}_{\alpha}(с \oplus \beta), с \in GF(2^m)\} = \{1 + (2^m + 1)\gamma(1), \ldots, 2^m + (2^m + 1)\gamma(2^m)\}.
\]
\end{proof}

\textbf{Теорема 2}\textit{(TU-разложение TKlog).}
Перестановка \(\mathscr{T}_{\kappa, s}\) над \(GF(2^{2m})\) имеет TU-разложение, включающее три \(m\)-битные перестановки \(\tau, \nu\) и \(\sigma\), и \(m\)-битную функцию \(\phi\). Это представлено в Алгоритме \ref{alg:alg02}.

\begin{algorithm}[htp!]
  \KwData{$x \in \text{GF}(2^8)$}
  $a, b \gets Split_{\beta}(x)$ \Comment*[r]{Входной линейный слой}
  \eIf{$a = 0$} {
    $\ell \gets \tau(b)$;
  }{
    $\ell \gets \nu(b/a)$;
  }
  $h \gets \sigma(\phi(\ell) \odot a)$ \Comment*[r]{$h=U_{T_a(b)}(a)=U_{\ell}(a)$}
  return $k(\ell) \oplus h$ \Comment*[r]{Выходной линейный слой}
\caption{TU-разложение TKlog}
\label{alg:alg02}
\end{algorithm}

\begin{proof}
Пусть \((a, b) = \mathrm{Split}_{\beta}(x)\). Пусть \(a\) будет соответствовать правой ветви TU-разложения, а \(b/a\) левой. Случай \(a = 0\) является особым случаем. Это доказательство является прямым в том смысле, что оно заключается только в правильном определении подкомпонентов \(\tau, \sigma, \nu\) и \(\phi\) и проверке их работы.

$\bm{a=0.}$ Если \(a = 0\), то \(x \in GF(2^m)\), и по определению TKlog получаем \(\mathscr{T}_{\kappa, s}(x) = \kappa(\tau(b))\), где
\[
\tau : 
\begin{cases}
GF(2^m) & \to \mathbb{F}^m_2 \\
0 & \mapsto 0, \\
b & \mapsto 2^m - \mathrm{log^{FLY}}_{(\alpha^{2m+1})}(b), \text{ если } b \neq 0.
\end{cases}
\]

Действительно, пусть \(x = b = \alpha^{(2^m+1)j}\) для \(0 < j \leq 2^m - 1\). Для \(b \neq 0, 1\) получается, что \(\mathrm{log^{FLY}}_{(\alpha^{2^m+1})}(b) = \mathrm{log}_{\alpha}(b)/(2^m + 1) = j\), тогда \(\kappa(\tau(b))\) равно \(\mathscr{T}_{\kappa, s}(x)\). Для \(b = 0\), получаем, что \(\kappa(\tau(b)) = \mathscr{T}_{\kappa, s}(x)\). Для \(b = 1\), \(\mathrm{log^{FLY}}_{\alpha}(b) = 2^{2m} - 1 = (2^m - 1)(2^m + 1)\), тогда \(j = 2^m - 1\), что равно \(\mathrm{log^{FLY}}_{(\alpha^{2^m+1})}(b)\).

$\bm{a \neq 0.}$ В противном случае, для \(a \neq 0\), можно записать \(x = \alpha(\beta \oplus b/a)\) и, поскольку 0 и 1 являются элементами \(GF(2^m)\), получается, что \(x \neq 0, 1\), тогда \(\mathrm{log^{FLY}}_{\alpha}(x) = \mathrm{log}_{\alpha}(x)\). Для оценки \(\mathscr{T}_{\kappa,s}(x)\), необходимо найти \(i\) и \(j\) такие, что \(\alpha(\beta \oplus b/a) = \alpha^{i+(2^m+1)j}\). Получается, что \(\mathrm{Tr}_m(a) = 0\) и \(\mathrm{Tr}_m(\beta \oplus b/a) = 1\), поэтому можно применить Лемму 2, чтобы записать
\begin{equation}
  \begin{cases}
    \beta \oplus b/a & = \alpha^{i+(2^m+1)\gamma(i)} \\
    a & = \alpha^{(2^m+1)(j-\gamma(i))}.
  \end{cases}
  \label{eq:03}
\end{equation}

Определим перестановки \(\nu, \sigma\) и функцию \(\phi\) следующим образом:

\begin{itemize}
  \item Перестановка \(\nu\) отражает то, как логарифм и арифметическая операция \(x \mapsto 2^m - x\) действуют на \(b/a\), чтобы вернуть правильный вход для \(\kappa\):
  \[
  \nu :
  \begin{cases}
  GF(2^m) & \to \mathbb{F}^4_2 \\
  c & \mapsto 2^m - (\mathrm{log}_{\alpha}(\beta \oplus c) \mod (2^m + 1)),
  \end{cases}
  \]
  тогда \(\nu(b/a) = 2^m - i\).

  \item Перестановка \(\sigma\) соответствует перестановке \(s\), примененной к \(j\):
  \[
  \sigma :
  \begin{cases}
  GF(2^m) & \to GF(2^m) \\
  0 & \mapsto 0 \\
  c & \mapsto (\alpha^{2^m+1})^{s(\mathrm{log}_{(\alpha^{2^m+1})}(c))}, \text{ при } c \neq 0.
  \end{cases}
  \]

  \item Функция \(\phi\) соответствует функции \(\gamma\), введенной в Лемме 2:
  \[
  \phi :
  \begin{cases}
  \mathbb{F}^m_2 & \to GF(2^m) \\
  i & \mapsto \alpha^{(2^m+1)\gamma(2^m-i)},
  \end{cases}
  \]
  где \(i \in \mathbb{F}^m_2\) интерпретируется как элемент \(\mathbb{Z}/(2m + 1)\mathbb{Z}\). Стоит заметить, что \(\phi(i) \neq 0\) для любого \(i\). Используя \(\nu(b/a) = 2^m - i\), получается:
  \[
  (\phi \circ \nu)(b/a) = \phi(2^m - i) = \alpha^{(2^m+1)\gamma(i)}
  \]

  и, также используя уравнение~\ref{eq:03}, получаем
  \[
  (\phi \circ \nu)(b/a) \odot \alpha = \alpha^{(2^m+1)\gamma(i)} \odot \alpha^{(2^m+1)(j-\gamma(i))} = \alpha^{(2^m+1)j}.
  \]
\end{itemize}

  Объединяя полученный результат с определением \(\sigma\), получаем
  \[
  \sigma((\phi \circ \nu)(b/a) \odot \alpha) = \sigma(\alpha^{(2^m+1)j}) = (\alpha^{2^m+1})^{s(j)}.
  \]

Тогда можно записать для любого \(a \neq 0\), что
\[
  \mathscr{T}_{\kappa, s}(a \beta \oplus b)=\kappa(\underbrace{\nu(b / a)}_{2^m-i}) \oplus \underbrace{\sigma((\phi \circ \nu)(b / a) \odot a)}_{\left(\alpha^{2^m+1}\right)^{s(j)}} .
\]

Когда \(a = 0\), элемент справа от \(\oplus\) аннулируется, так как \(\sigma(0) = 0\). Поэтому, для случая \(a = 0\), можем записать:
\[
  \mathscr{T}_{\kappa, s}(a \beta \oplus b)=\underbrace{\kappa(\tau(b))}_{\mathscr{T}_{\kappa, s}(b)} \oplus \underbrace{\sigma((\phi \circ \nu)(b / a) \odot a)}_0,
\]
что представляет собой то же выражение, что и в случае \(\alpha \neq 0\), за исключением того, что аргумент \(\kappa\) принимает значение \(\tau(b)\) вместо \(\nu(b/a)\). В результате можно оценить эту функцию в два этапа:

\begin{enumerate}
  \item Пусть \((a, b) = \mathrm{Split}_{\beta}(x)\). Если \(a = 0\), пусть \(\ell = \tau(b)\); иначе пусть \(\ell = \nu(b/a)\).
  \item Вернуть \(\kappa(\ell) \oplus \sigma(\phi(\ell) \cdot \alpha)\).
\end{enumerate}

Этот процесс описан в Алгоритме~\ref{alg:alg02}.
\end{proof}

Существование TU-разложения Бирюкова и др., очевидно, является прямым следствием этой теоремы. Связь между этим разложением и исходной структурой TKlog позволяет лучше понять некоторые закономерности, существующие в этом разложении.

Во-первых, компонент, который Бирюков и др. назвали \(\nu_0\), играет роль \(\tau\) в Теореме 2. Поскольку \(\tau\) по сути является логарифмом с основанием \(\alpha^{-(2m+1)}\), неудивительно, что в \cite{PU16} было отмечено, что \(\nu_0\) аффинно-эквивалентен логарифму.

Во-вторых, линейные слои \(\alpha\) и \(\omega\) исходного разложения по сути разбивают \(GF(2^{2m})\) на \(GF(2^m)^2\), а затем соединяют обратно, отсюда и отношение, представленное в Уравнении~\ref{eq:01}. Кроме того, выходной слой \(\omega\) сложнее входного слоя \(\alpha\), поскольку он также оценивает линейную часть \(\kappa\).

\subsection{Свойства, объясняемые TU-разложением}

\textbf{Эффективная аппаратная реализация.} Знание разложения из Теоремы 2 позволило Бирюкову и др. значительно улучшить реализацию \(\pi\): они сделали схему, реализующую данную перестановку, уменьшив площадь и задержку в 2.5 и 8 раз соответственно \cite{BPU16b}. Можно ожидать аналогичных улучшений для всех 8-битных TKlog.

\textbf{Визуальный артефакт в LAT.} Бирюков и др. использовали особый визуальный артефакт в LAT \(\pi\) для выполнения первого шага своего TU-разложения, а именно наличие столбцов с меньшим количеством различных коэффициентов, чем у других. Они показали, что эта закономерность является прямым следствием структуры, которую они определили в этом S-блоке, а именно структуры из Теоремы 2. Как следствие, можно обобщить их результат на все экземпляры TKlog.

\textbf{Лемма 3.}
Рассмотрим экземпляр TKlog \(\mathscr{T}_{\kappa,\alpha,s}\), пусть \(\text{Split}_{\beta}, \tau, \sigma, \nu\) и \(\phi\) будут подфункциями, используемыми для реализации его TU-разложения, как представлено в Алгоритме 2, и пусть \(\phi_{\kappa} : GF(2^{2m}) \to GF(2^m) \times GF(2^m)\) будет аффинной биекцией вида \(\phi_{\kappa}(\kappa(x) \oplus v) = (x, v)\).

Пусть \(F\) перестановка над \(GF (2^m)^2\), определенная как \(F = \phi_{\kappa} \circ \mathscr{T}_{\kappa,s} \circ \text{Split}^{-1}_{\beta}\), она состоит из TU-разложения \(\mathscr{T}_{\kappa,s}\) без входных и выходных линейных слоев. Тогда, для \((a_L, a_R) \neq (0, 0)\) и \(b_L \neq 0\) получается
\[
\mathcal{W}_{F}[(a_L, a_R), (b_L, 0)] = \left(\mathcal{W}_{\tau}[a_L, b_L] \pm 2^m\right)[a_L \neq 0],
\]
где \([a_L \neq 0] = 0\) при \(a_L = 0\), и \(1\) в противном случае.

Доказательство этой леммы по сути аналогично тому, которое дали Бирюков и др. в расширенной версии своей статьи \cite{BPU16b}. Для полноты изложения, оно будет приведено в Приложении Г. 

Поскольку \(\tau\) по сути является дискретным логарифмом, его нелинейность всегда велика, так что закономерность всегда видна. Действительно, из-за этой высокой нелинейности, \(|\mathcal{W}_{\tau}[a_L, b_L]|\) может принимать значения, значительно меньшие, чем \(2^m\), что означает, что низкий контраст будет всегда присутствовать.

В работе \cite{BPU16a} авторы оригинального разложения отметили, что им повезло, поскольку стечение обстоятельств, которое привело к успеху их разложения, казалось маловероятным. По их собственным словам (страница 19 из \cite{BPU16b}):
\begin{quote}
[Визуальный шаблон] вызван совокупностью трех элементов:
\begin{itemize}
\item использование мультиплексора;
\item использование инверсии в конечном поле;
\item факт того, что \(\nu_0\) обладает хорошей нелинейностью
\end{itemize}
По иронии судьбы, единственная ''неудивительная'' подкомпонента \(\pi\), а именно обратная функция, является одной из причин, по которой нам удалось осуществить реверс-инжиниринг именно этого S-блока. Если бы [инверсия] была заменена на другой (и, возможно, более слабый!) S-блок, не было бы ни одной из строк в LAT, с которых начался бы наш реверс-инжиниринг.
\end{quote}

Теперь видно, что все эти события являются не удачей, а прямым следствиемс структуры TKlog \(\pi\).

Кстати, это отличный метод для выявления такой структуры, если кто-то другой решит использовать ее и попытается скрыть этот факт. В работе \cite{PU16} Перрин и Удовенко предложили построить график дисперсии абсолютного значения коэффициентов в каждой строке/столбце LAT. Они заметили, что для \(\pi\) в некоторых столбцах наблюдается резкое падение. Из Леммы 3 следует, что эта закономерность является неотъемлемым свойством TKlog, которое всегда будет свидетельствовать о наличии такой структуры. Более того, если экземпляр TKlog усложнить (obfuscated), составив его из аффинных слоев, этот шаблон сохранится и фактически даст некоторую информацию о линейной части этих аффинных слоев.

Кроме того, поскольку TKlog всегда имеет TU-разложение, он всегда будет иметь векторное пространство размерности \(n = 2^m\)  внутри множества координат нулей его LAT \cite{CP19}. Это также можно обнаружить, и таким образом выяснить, что экземпляр является TKlog.

\textbf{CCZ-эквивалентность.} Из Теоремы 2 известно, что TKlog всегда аффинно-эквивалентен перестановке \(P : (GF(2^m))^2 \to (\mathbb{F}_2^m)^2\), такой, что
\[
P(b, a) = \left(T_a(b), U_{T_a(b)}(a)\right), \text{ где }\left\{\begin{array}{l}
T_a(b) = \tau(b) \cdot [a = 0] \oplus \nu(b/a) \times [a \neq 0], \\
U_k(a) = \sigma(a/\phi(k)),
\end{array}\right.
\]
где \([a = 0]\) — это булева функция, отображающая \(0\) в \(1\) и \(a \neq 0\) в точке \(0\), и где \([a \neq 0] = 1 \oplus [a = 0]\).

Как объясняется в \cite{CP19}, существование такого разложения, в котором \(T_r\) является перестановкой для всех \(r \in \mathbb{F}_2^m\), эквивалентно возможности так называемого \(m\)-поворота (\(m\)-twist), операции, которая сохраняет класс CCZ-эквивалентности, но априори не сохраняет класс AE-эквивалентности. Поскольку функциональная инверсия также сохраняет класс CCZ-эквивалентности, получается следующее следствие из Теоремы 2.

\textbf{Следствие 3.}
Пусть \(T\) и \(U\) такие, как определено выше. И пусть соответствующий экземпляр TKlog и его инверсия являются CCZ-эквивалентными функции \(F : \mathbb{F}_2^m \times GF(2^m) \to \mathbb{F}_2^m \times GF(2^m)\), такой, что
\[
F(b, a) = \left(T_a^{-1}(b), U_b(a)\right), \text{ где }\left
\{\begin{array}{l}
T_a^{-1}(b) = \tau^{-1}(b) \times [a = 0] \oplus \left(\nu^{-1}(b) \odot a\right) \times [a \neq 0], \\
U_b(a) = \sigma(a/\phi(b)),
\end{array}\right.
\]
\begin{proof}
По определению (\cite{CP19}), \(m\)-поворот отображает функцию \(P : (r, \ell) \mapsto \left(T_r(\ell), U_{T_r(\ell)}(r)\right)\) в \(F : (r, \ell) \mapsto \left(T_r^{-1}(\ell), U_{\ell}(r)\right)\).
\end{proof}

В случае \(\pi\) получилось сгенерировать функцию \(F_\pi\), которая является CCZ-эквивалентной ей, как указано в Следствии 3. Ее таблица поиска представлена в Приложении А (см. Таблицу \ref{tab:A2}). Она имеет Такие же разностные и расширенные спектры Уолша, что и \(\pi\), и, как и \(\pi\), все ее координаты имеют степень 7. Однако это не перестановка: у 15 элементов есть по 3 прообраза, у 75 — по 2, а у оставшихся 61 — по 1.