\abstract

Стрибог и Кузнечик — это симметричные криптографические примитивы, стандартизированные российским ГОСТ. Они используют один и тот же S-блок \(\pi\), процесс генерации которого не был описан его авторами. В предыдущих работах Бирюков, Перрин и Удовенко восстановили два совершенно разных разложения этого S-блока. В этой работе будут пересмотрены их результаты и будет представлено третье разложение S-блока \(\pi\). Оно является примером довольно небольшого семейства перестановок, оперирующих \(2m\) битами, которое называется TKlog и которое тесно связана с логарифмами в конечных полях. Его простота и малое количество используемых компонентов позволяют утверждать, что это та структура, которая была намеренно использована разработчиками Стрибог и Кузнечик.

\(2m\)-битные перестановки этого типа обладают очень сильной алгебраической структурой: они отображают мультипликативные смежные классы подполя \(\text{GF}(2^m)^*\) в аддитивные смежные классы \(\text{GF}(2^m)^*\). Более того, функция, связывающая каждый мультипликативный смежный класс с соответствующим аддитивным смежным классом, всегда по сути одна и та же. Насколько известно, в данной работе впервые представлена эта очень сильная алгебраическая структура.

В данной работе также исследуются другие свойства TKlog и показываются, в частности, что он всегда может быть разложен подобно первому разложению Бирюкова и др., что объясняет связь между двумя предыдущими разложениями. Это также означает, что всегда возможно эффективно реализовать TKlog аппаратно, и что он всегда демонстрирует визуальный паттерн в своей LAT, аналогичный тому, который присутствует в \(\pi\).

Хотя в рамках данной работы не были найдены атаки, основанные на этих новых результатах, тут обсуждается влияние этой работы на безопасность Стрибог и Кузнечик. Для этого было предоставлено новое, более простое представление линейного слоя Стрибог в виде умножения матриц в точно таком же поле, в каком определена \(\pi\). Из-за чего можно сделать вывод, что эта матрица взаимодействует нетривиальным образом с разбиениями, сохраняемыми \(\pi\).
