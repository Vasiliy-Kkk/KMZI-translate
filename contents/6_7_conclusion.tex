\section{Заключение}
Мы извлекли новую структуру из \(\pi\), которую утверждаем как изначально задуманную её разработчиками. Её обобщение, TKlog, получается путём композиции дискретного логарифма с простым слоем арифметики. TKlog объясняет обе предыдущие декомпозиции \(\pi\), таким образом обеспечивая недостающее звено между этими двумя результатами. Знание об этой декомпозиции позволило нам объяснить очень специфическое свойство сохранения разбиения \(\pi\). Удивительно, но мы также нашли новое выражение линейного слоя Стрибога, выраженное в том же конечном поле как и \(\pi\). Хотя мы не можем использовать эти свойства для атаки на эту хеш-функцию, мы ставим под вопрос уместность этого проектного решения. Действительно, при работе с компонентами, определёнными в идентичных математических структурах, академические разработчики нарушают это соответствие, например, составляя их S-блоки с несвязанными аффинными перестановками. Мы склонны считать, что было бы более осторожно поступить так же в Стрибог.

\section{Благодарности}
Мы благодарим рецензентов IACR ToSC за их внимательное чтение и комментарии. Также благодарим Анну Кантаут за замечания, которые значительно улучшили презентацию этой работы. Работа автора была профинансирована грантом для пост-докторантов от Fondation Sciences Mathématiques de Paris (FSMP).
